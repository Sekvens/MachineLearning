\documentclass[a4paper]{article}

\usepackage[english]{babel}
\usepackage[utf8]{inputenc}
\usepackage{amsmath}
\usepackage{listings}
\usepackage{color}
\usepackage{hyperref}
\usepackage{float}

\title{Machine Learning (course 1DT071)
Uppsala University – Spring 2015
Report for Assignment 2 by group 6}

\author{Ludvig Sundstr\"{o}m and John Shaw}
\date{\today}

\begin{document}

\maketitle

\section{Task 1: Q-learning}

\subsection*{Question 1}
\emph{After the arrows have converged, some states will have longer
arrows than others (i.e., larger Q-values). Why is this so?}

\textbf{Answer:} The discount factor restricts the values to become up to 0.95 times the value of a optimal future state. Because of this, the maximum Q-value for a given state will be smaller the further from the goal it is. 

\subsection*{Question 2}
\emph{In the first few episodes, only the states closest to the goal will
have their Q-values increased (even though the agent may start far from the
goal). Explain why.}

\textbf{Answer:} Initially, all Q-values are zero. Before the agent finds the goal, no reward is given and hence no values are changed. With the reward, state 7 has it's value changed. In the following epochs, actions that takes the agent to this state will have their value changed. Therefore the Q-values will start increasing at states closer to the goal.

\pagebreak
\subsection*{Question 3}
\emph{Certain arrows (not necessarily in the same state) should con-
verge towards the same lengths. Why is this the case? Give an example of two
state-action pairs that get equal Q-values.}


\textbf{Answer:} Because the map is symmetrical, some states are also symmetrical. For example states \{10, 14\} are two states 4 cells from the goal where it's only possible to move North or South. In both cases, both states will take the agent equally close to the goal (3 cells). 

\subsection*{Question 4}
\emph{If you train the agent long enough, the red arrows will mark the
shortest path to the goal. Why does it find the shortest path? (You may have noticed that the problem formulation—how the agent is rewarded—did not define shorter paths as ``better''.)}

\textbf{Answer:} The action that takes the robot closer to the goal gets higher Q-values because the way Q-learning algorithm works. The shortest path is found because of the discount factor will cascade so more distant states get a smaller proportion of the reward. When the number of episodes get higher we can assume it have started from all locations in a even distribution and that it have taken all paths in a even distribution. Then the red arrows will reflect the impact the discount factor have had to all other paths, making the red arrow the most rewarded path.
During an update the discount factor have lowered the values of a sub-optimal path's state. When updating the part of the rewards it will become lower than if it was coming from an optimal path. 

\subsection*{Question 5}
\emph{How can we reformulate the problem (change the immediate re-
wards that the agent receives) so as to explicitly make shorter paths preferable
to longer paths, also if the metric of success is simply the total sum of 
rewards in an episode?}
\\\\
\textbf{New answer:} A simpler solution would be to punish the agent every
time an action does not bring it to the goal. This way, we would not need to
know anything about the solution before training. 
\\\\
\textbf{Answer:} One solution is to punish the poor robot if it moves away from 
the goal and reward it if it walks closer. 
Another alternative could be to calculate the Manhattan distance (since we know 
how this environment is constrained) and reward it more when it's closer to the 
Manhattan distance and less if it's moving away from the Manhattan distance.

\subsection*{Question 6}
\emph{Write down the update rule for Q-learning!}

\textbf{Answer:} 
The update rule is the TD error:
\begin{equation*}
Q(s, a) \leftarrow Q(s, a) + \eta \bigg( r + \gamma max(Q(s', a') - Q(s, a)) \bigg)
\end{equation*}

 where $r$ is the reward, $\eta$ is the learning rate and $\gamma$ is the discount factor.

\subsection*{Question 7}
\emph{What is the true value (the value that Q-learning should converge
to, Q*) of the action down in state 2 (top row, centre square)? Include the complete calculation, not just the answer.}

\textbf{Answer:} The true value for state 2 is 0.95.

\begin{align*}
Q(s, a)                    &= \\
r + \gamma \times max(Q(s', a'))   &= \\
0 + 0.95 \times \sim lim(1) &= 0.95 \\
\end{align*}

\subsection*{Question 8}
\emph{Some arrows that previously converged to equal length (with $\epsilon = 1.0$) will not any more (unless you train for a very long time). Why?}

\textbf{Answer:} Because of greediness the algorithm is much more likely to visit a state that has sooner or later led to the goal. This means that even though there might exist more optimal solutions, the agent are much more likely to take a known path rather than trying a new undiscovered path. 
This leads to states that are far away from the optimal paths are less likely to get visited and thus updated less frequently.

\section{Task 2: SARSA}

\subsection*{Question 9}
\emph{With Q-learning the lengths of the arrows increased steadily, but
using SARSA they sometimes decrease. Why is that?}

\textbf{New Answer:} Unlike Q-learning, SARSA does not always update the
Q-value towards the maximum Q-value of the next state. Instead it updates
towards the Q-value of the selected action in the next state. If that Q-value
happens to be smaller than the Q-value of the current state, it will result in
a negative update, and the arrows in the simulation will shrink.
\begin{equation*}
Q(s, a) \leftarrow Q(s, a) + \eta \bigg(\overbrace{ r + \gamma Q(s', a') - 
Q(s, a)}^\text{Becomes less than 0} \bigg)
\end{equation*}
\\\\
\textbf{Answer:} When Q-learning selects an action it directly update the 
previous state. SARSA however, selects one action and then explores it. 
The previous state is not necessary updated right away and SARSA can 
select another action if it's not satisfied with the explorations of the 
first chosen action. This makes the Q-values of an action getting updated 
(the length of the arrows increase) while SARSA is exploring, however when 
SARSA decides to explore another action it restores the Q-values and thus 
the arrows of the first explored action decrease.

When trying to visualize the agent for the two algorithms one difference 
would be that SARSA have many actions to choose from while Q-learning 
just takes the one that maximise for the moment (greediness).

\subsection*{Question 10}
\emph{Do the arrows in the SARSA window converge to the same
lengths as the corresponding arrows in the Q-learning window? Motivate your answer.}

\textbf{Answer:} Yes, if given enough time they will. SARSA and Q-learning are very similar algorithms and they differ in the way they update on exploration. If given enough time they will have explored all possible states and actions so the difference between them can be ignored. 

Since they calculate the Q-values in the same way they will eventually converge into the same value.

\section{Task 3: Two rooms}

\subsection*{Question 11}
\emph{What can you say about the average action count in the experiment? Compare guiding the robot to not guiding the robot.}

\textbf{Answer:} 
The number of actions needed to finish 50 episodes decreases heavily if the agent is guided the first few episodes. Initially robot has no information about its environment and has to learn by trial and error. It requires a large number of random actions before it starts learning how to get closer to the goal. If one experiences that the agent learns too slow in the beginning, one can guide the robot a couple of times, showing it the closest path to the goal. Now when the robot has an idea on where to go, it should speed up the learning process. 

\subsection*{Question 12}
\emph{In general, can you think of any disadvantages of guiding the
agent in this way?} 

\textbf{Answer:} If you manually guide the agent too much the agent might get ``tunnel vision'' and always perform the guided solution. This would defeat the purpose of reinforcement learning where we only want to inform the agent when it has done something good or bad rather than telling the agent exactly what it did that was good or bad. If it's allowed to explore in the start it could find very innovative starts and guiding it could make it require a lot of exploration to find the same innovative starts. Another problem arises if the environment changes state. A wall might appear in the way of the robots guided solution making it very hard for it to adapt. 

\subsection*{Question 13}
\emph{Suggest an application where this method of initially leading the
agent may be necessary (if not in theory, then at least in practice).}

\textbf{Answer:} If the state space is large and it's an constraint optimization problem, for example scheduling, there might exist-non optimal solutions that are easy to find. By guiding the agent in the start it would not wander aimlessly around in the state space but could instead start exploring solutions that are similar to the guided solution. 

When trying to make a robot learn how to balance a pole or similar guiding could help the application to find a start, avoiding to much exploring in states where the stick is held vertically or other extremes.

\end{document}
