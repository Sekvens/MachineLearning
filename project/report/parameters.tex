\section{Parameters}
After browsing through a list of spam messages we concluded that it was common to split links with whitespaces in spam messages as an attemt to avoid spam detection. Links seemed to be more common in spam messages than in ham. Because of that we made a new parameter that checks if the message contains any of the words \texttt{html, http, www TODO} and we use that as a parameter for the ANN. 

Another interesting hypothesis we had was that using a word list of words that are more common in spam messages could improve the performance of the spam detection. We composed a simple word list containing the words \texttt{win, secret} that would be used as a parameter.
These parameters are binary and true if they contain any of the words in these lists. We didn't want to use a counter since it would discriminate against ham messages that are written about a subject that might contain many of the banned words and therefore increase the risk of a faulty classification. 

One problem with the word list is that spammers might change their vocabulary to avoid our spam detection and thus decrease the chance of spam detection and might increase the risk of faulty classification of hams as spam. 

After running our tests we can see that the word list with the banned words are greatly increasing the chance of classifying spam messages correctly. The word list that identifies links have a much smaller impact but improves the spam classification slightly. 

Results:
\begin{table}[h]
\begin{tabular}{llll}
Run               & Overall Success & Spam Success & Ham Success \\
Both Lists        & 84.24\%         & 79.37\%      & 83.07\%     \\
Banned Word List  & 86.37\%         & 67.31\%      & 92.75\%     \\
Link Word List    & 82.86\%         & 68.34\%      & 86.50\%     \\
None of the Lists & 83.83\%         & 48.66\%      & 96.01\%    
\end{tabular}
\end{table}