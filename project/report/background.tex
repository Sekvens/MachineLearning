
\section{Background}

Ever since the birth of electronic communication, its users have been exposed
to spam messages. Of all the possible forms of communication, e-mail are
generally the most common way of delivering spam. 
According to a recent report from M\textsuperscript{3}AAWG percentage of
e-mails that they classified as abusive between 2012-2014 have been between
$87-90\%$ \cite{M3AAWG2014}.  However, spam appears in many other forms of
communication like SMS text messages, internet chats, social media and many
more. The purpose of most spam is to advertise a product in a cheap way to a
large number of recipients. Since spam advertisement is considered unethical
most companies would not want to be associated with spam.  Therefore it's
generally more shady products that are being advertised like diets, hoax
products, pornography, escort services or pharmaceutical products from
unreliable distributors\cite{wikipedia}. Some of these products can be
dangerous for the recipients and there are as well spam that are purely
intended to harm the recipient through scams or with the intent of stealing or
hijacking information.  Another issue with spam is that it's wasting the
recipients time and bandwidth. Currently there are many organizations that are
working with preventing spam reaching the recipient.  Justin M. Rao and David
H. Reiley \cite{rao2012economics} have estimated that the gross revenue of the
worldwide spam advertisement is in the order of \$200 million per year and this
number would be far greater if it wasn't for the current prevention efforts.
The most efficient way to stop spam is to create programs that are called
anti-spam filters. Black-lists where a common anti-spam filter technique which
have had a limited effect since spammers usually change identity on the
internet.  Context-based filters was another common technique that is searching
for patterns common in spam messages, the drawback with these technique are
that these patterns have to be hand written and that the spammers usually adapt
their spam messages to avoid detection. New techniques are developed to tackle
the ever evolving spam messages.\cite{spam-techniques} Some spam filtering
systems take advantage of Machine-learning methods to improve their accuracy. A
very popular spam-filtering technique that is sometimes classified as a machine
learning technique is the Naive Bayes classifier \cite{bayes}, which operates
on bags on words to identify spam mails. Because it operates on words, it is
susceptible to Bayesian Poisoning, which is a technique used by spammers to get
around these filters. A spammer practicing this technique will send a large
amount of text taken from legitimate messages together with the spam message in
order to confuse the classifier.  \\\\ If a message clobbered with a lot of
nonsense makes it easy for most people to decide that its a suspicious message.
In fact, most people are usually able to decide very fast weather a message is
spam or not no matter what structure the message has. However, if one would ask
them how they decide that, the answer is not always obvious.  
ANN's has the same principles - once a network is trained, it can
tell us of what class a certain pattern belongs to, but it's very hard to study
how the system actually classifies that specific pattern. This leads us to
believe that the problem domain fits a ANN well. Perhaps the most useful thing
about the problem with studying the classification is that if we can't study
exactly how the network classifies the messages spammers won't be able to do it
as well. As a result, they won't be able to develop spam techniques to use
against the ANN classifier.
