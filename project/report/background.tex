
\section{Background}

Ever since the internet was born, users have been exposed to spam
messages. According to a recent report from M\textsuperscript{3}AAWG percentage
of e-mails that they classified as abusive between 2012-2014 have been between
$87-90\%$ \cite{M3AAWG2014}. Spam appears in many other forms of communication
like SMS text messages, internet chats, social media and many more.  The
purpose of most spam is to advertise a product in a cheap way to a large number
of recipients. Since spam advertisement is considered unethical most companies
would not want to be associated with spam.  Therefore it's generally more shady
products that are being advertised like diets, hoax products, pornography,
escort services or pharmaceutical products from unreliable distributors. Some
of these products can be dangerous for the recipients but there are as well
spam that are purely intended to harm the recipient through scams or with the
intent of stealing or hijacking information.  Another issue with spam is that 
it's wasting the recipients time and bandwidth. Currently there are many
organizations that are working with preventing spam reaching the recipient.
Justin M. Rao and David H. Reiley \cite{rao2012economics} have estimated that
the gross revenue of the worldwide spam advertisement is in the order of \$200
million per year and this number would be far greater if it wasn't for the
current prevention efforts. The most efficient way to stop spam is to create
programs that are called anti-spam filters. Black-lists where a common
anti-spam filter technique which have had a limited effect since spammers
usually change identity on the internet. Context-based filters was another
common technique that is searching for patterns common in spam messages, the
drawback with these technique are that these patterns have to be hand written
and that the spammers usually adapt their spam messages to avoid detection.
New techniques are developed to tackle the ever evolving spam messaging
techniques.\cite{spam-techniques} Many spam filtering systems take advantage of
Machine-learning methods to improve their accuracy. \\\\
Most people can decide very fast weather a message is spam or not, but if one
would ask them how they decide that, the answer is not always obvious. 
ANN's work in a similar way - once a network is trained, it can tell
us of what class a certain pattern belongs to, but it's very hard to study how
the system actually classifies that specific pattern. This leads us to believe
that the problem domain fits a neural network well. 
