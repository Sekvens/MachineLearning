
\section{Methods}
  
  The goal for the practical part was to try to train an ANN on a reliable
  dataset and make it correctly classify as many messages as possible,
  preferably the entire set.
  Acquiring a reliable data set to train on was surprisingly easy. An
  example of an available dataset is the Enron data set \cite{enron} which was
  published to be available for research purposes. Using this dataset, 
  we used Node.js to get the main content string from the
  message-file, discarding any meta data and other uninteresting characters.
  That is to say removing HTML tags, subject fields etc. The strings were then
  lemmatized with the python NLTK lemmatizer \cite{nltk} and
  \cite{bird2009natural}. Once obtained the preprocessed, pure content strings
  we used a Haskell program to perform the
  actual feature extraction. The features were written as line separated floats
  to one file per preprocessed message.
  \subsection{Initial choice of feature vector}
    The initial list of features that we extracted were chosen as follows:
    \begin{poem}\mbox{}\\[-\baselineskip]
      \begin{enumerate}
        \item Total number of characters
        \item Total number of white characters 
        \item Total number of alphabetic characters
        \item Total number of digit characters
        \item Total number of punctuation characters
        \item Ratio between alphabetic characters and total number of characters
        \item Total number of words
        \item Total number of words with less than 3 characters
        \item Average word length 
        \item Average sentence length in words
        \item Average sentence length in characters
        \item Frequency of words of length 1
        \item Frequency of words of length 2
        \item Word diversity index, Simpsons D measure \cite{simpsons-measure}
        \item Value depending on if message contained a word defined in the
          predefined list of spam words
        \item Value depending on if message contained a word defined in the 
          predefined list of link words.
      \end{enumerate}
    \end{poem} 

    The two last features were an attempt to manually tweak the algorithm.
    For these features, we created two lists of words that we felt were more
    common in spam messages in a general sense. We called the lists "spam
    words" and "link words" where the former contained common spam words such
    as \texttt{win, reward} while the latter contained words such as
    \texttt{http, www}. The motivation for the second list came after we
    discovered spammers usually split hyperlinks in various ways.
    The two lists are presented in Appendix A. For each of the
    two lists, we set the corresponding feature to 1.0 if the
    message contained a word that the lists contained and 0.0 otherwise.
	
	This approach with lists of words is similar to naive algorithmic approaches 
	to spam detection where the performance have been quite bad. They are as well 
	partly used for the rule based knowledge engineering techniques \cite{machine-learning-methods-spam}.
	However it differs in our approach since we are only giving the information to an ANN as an extra parameter
	used for training. The ANN then decides how important this information is and adapt it accordingly
	which differs from the naive algorithms where the word list would be what decides if it spam or not.
    One thing to keep in mind when "hard-coding" a word list in the system is
    that spammers might change their vocabulary to avoid our spam detection and
    thus decrease the chance of spam detection.    
  \subsection{Creating and training a network}


    Matlab has a built in toolbox for simulating and training neural networks
    \cite{matlab-nn}. 
    This toolbox is highly configurable, so the next step was to find the
    optimal parameters for our problem. In order to do so, we needed to train
    an ANN with different parameters and compare it's outputs. 
    
    Using Matlab to parse the files produced by the preprocessors, we obtained
    a total of ~2500 feature vectors. This set was partitioned into a train and
    test partition. 70\% of vectors were assigned to the training set and the
    rest were assigned to the testing set. 35\% of the total amount of the
    feature vectors were from spam messages. The feature vectors was
    normalized to range [-1, 1] to fit the range of the hidden-layer
    activation function that we planned on using. The matrix of features became
    the input (patterns) to the neural network. A target vector for both the
    training and the testing dataset were defined as follows: For each message,
    the corresponding element of the target vector were set to 1.0 if the
    message were a spam message, and set to 0.0 otherwise. 
    After training a network using the training dataset, we used the
    test dataset to determine how well trained the system were. 
    For a given message, if the corresponding system output was greater or
    equal than 0.5 and the target were 1.0, we defined that as a successfully
    classified message. If the target were 0.0 for that message, we defined
    that as a miss-classified message. The reverse logic was used if the output
    was less than 0.5.
    For determining the performance of a fully trained network was the ratio
    between the number of miss-classified messages and the total number of
    messages.
    To attempt to find the optimal parameters, we created a Matlab script
    capable brute forcing through training with all possible combinations of
    all common parameters and output the configurations for networks with the
    highest score. Letting this script run for many hours, we finally
    established the following parameters optimal for the first dataset:
    \begin{itemize}
      \item Training algorithm: traingd (Gradient descent)
      \item Learning rate: 1.0
      \item Number of hidden layers: 1
      \item Number of nodes in hidden layer: 9
      \item Number of epochs: 900
    \end{itemize}
    As activation functions, hyperbolic tangent and logistic were used
    for the hidden layer and output layer respectively during the entire
    session.  Training with resilient back propagation has more arguments and
    took longer to test, but as it turns out Gradient descent performed better
    for this dataset. Results from this training session with the highest score
    is presented in section \ref{sec:initialresults}, both including the "spam
    words" and "list words" features and excluding.

  \subsection{Improving the feature vector}
    Realizing that the output of the system were in range [0, 1], we wanted to
    normalize each individual feature to that range. The next generation of
    features were became:
  \begin{poem}\mbox{}\\[-\baselineskip]
    \begin{enumerate} \label{reworked-features}
      \item Total number of words (normalized to [0, 1])
      \item Total number of words with less than 3 characters 
        (normalized to [0, 1]) 
      \item Average word length 
        (normalized to [0, 1]) 
      \item Average sentence length in words
        (normalized to [0, 1]) 
      \item Average sentence length in characters
        (normalized to [0, 1]) 
      \item Ratio alphabetic characters / total amount of characters
      \item Ratio digital characters / total amount of characters
      \item Ratio punctuation characters / total amount of characters
      \item Ratio whitespace characters / total amount of characters
      \item Frequency of words of length 1
      \item Frequency of words of length 2
      \item Word diversity index, Simpsons D measure \cite{simpsons-measure}
    \end{enumerate} 
  \end{poem} 
  
  Results from training with these features are presented in section
  \ref{sec:reworkedresults}.

  \subsection{Deciding the important features}
    Regardless of how a network performs, some features might be more important
    than others. To investigate this, we trained automated a script to train
    the network on different subsets of these features. When we found subsets
    that performed well, we gradually removed individual features, trained the
    network again and studied the result. The most important features are
    presented in section \ref{sec:important-features}.
